\documentclass[compress]{thesisbeamer}
\useoutertheme[subsection=false]{miniframes}
\usepackage{nccmath}
\usepackage{mathtools}
\usepackage[thinc]{esdiff}

\title{A Gazebo Simulator for Continuum Parallel Robots}
\author[Gotelli Andrea]{Author Gotelli Andrea\newline ~ \newline \normalsize{Advisors: Sébastien Briot, Olivier Kermorgant, Federico Zaccaria, Antonio Sgorbissa}}
\date{22/02/2021}

\bibliography{../biblio}

% video will work on Linux (with Okular) or OS X, for other OS's or viewers find your own way to do it
%\videoOSX	% for OS X 
%\videoOFF
\setbeamerfont{caption}{size=\tiny}

\begin{document}

\MakeTitleNoFoot

\begin{frame}
	\frametitle{Index}
	\begin{enumerate}
		\item Introduction to Continuum Parallel Robots (CPR)
   		\item Modelling of deformable links
   		\item Methods
   		\item Conclusions
	\end{enumerate}	
\end{frame}

    
   	\section{Introduction to CPR}	
        \subsection{Serial_parallel_robots}
        \begin{frame}
        	\frametitle{Serial and parallel robots}
            \begin{columns}
			\column{0.5\textwidth}
			\begin{itemize}%[<+->]
  				\item Serial robots
  				\begin{itemize}%[<.->]
   					\item Simpler and more used
   					\item Limited by weight and inertia
  				\end{itemize}\vfill
  				\item Parallel robots
  				\begin{itemize}%[<.->]
   					\item Less inertia, high speed
   					\item More joints involved
  				\end{itemize}\vfill
 			\end{itemize}
			\column{0.5\textwidth}
			\begin{figure}[h]
				\centering
				\includegraphics[width=\textwidth]{images/serial_parall_robots}
				\caption{Different robot architectures \cite{bryson_toward_2014}.}	
			\end{figure}
			\end{columns}
		\end{frame}

        \subsection{CPR}
        \begin{frame}
        	\frametitle{Continuum parallel robots}
			\begin{columns}
			\column{0.5\textwidth}
			\begin{itemize}%[<+->]
  				\item Advantages
  				\begin{itemize}%[<.->]
   					\item May enhance safety
   					\item Cheaper components 
   					\item Possible to miniturize
  				\end{itemize}
  				\item Drawbacks
  				\begin{itemize}%[<.->]
   					\item More unstable configurations
   					\item Lower payload
   					\item Not analytical solution
  				\end{itemize}
  				\item Intention
  				\begin{itemize}
  					\item General simulator
  					\item Gazebo plugins
  				\end{itemize}
 			\end{itemize}
			\vspace{2cm}
			\column{0.5\textwidth}
			\begin{figure}[h]
				\centering
				\includegraphics[width=\textwidth]{images/serial_parall_robots}
				\caption{Different robot architectures \cite{bryson_toward_2014}.}
			\end{figure}
			\end{columns}
		\end{frame}

  	\section{Modelling of Continuum}
       	\subsection{Geometric_modelling}
        \begin{frame}
        	\frametitle{Geometric modelling: Cosserat rod thoery}
			\begin{itemize}%[<+->]
  				\item Rod as 1D body 
  				\item Function of the arc-lenght $ s $
  				\begin{itemize}%[<.->]
   					\item Centerline position $ p_{(s)}\in\mathbb{R}^3$
   					\item Cross-section orientation $R_{(s)}\in\textit{se}(3)$
  				\end{itemize}
  			\end{itemize}
  			\begin{columns}
			\column{0.5\textwidth}
			\begin{itemize}%[<+->]
  				\item Define transformation
  				\begin{fleqn}
  				\begin{equation}
  					T_{(s)} = 
  					\begin{bmatrix}
  						R_{(s)} & p_{(s)} 	\\
  							0	&	1		\\
  					\end{bmatrix}\in SE(3)
  				\end{equation}
  				\end{fleqn}
  				\item Derivative wrt arc-lenght $ x^{'} = \diff{x}{s} $
 			\end{itemize}
			\vspace{2cm}
			\column{0.5\textwidth}
			\begin{figure}[h]
				\centering
				\includegraphics[width=\textwidth]{images/rod_geometry}
				\caption{Rod geometric modelling \cite{till_statics_2019}.}
			\end{figure}
			\end{columns}
		\end{frame}

        \subsection{Rod_equilibrium}
        \begin{frame}
        	\frametitle{Equilibrium Equations}
			\begin{columns}
			\column{0.5\textwidth}
			\begin{itemize}%[<+->]
  				\item Equilibrium consideration
  				\begin{itemize}%[<.->]
   					\item Distributed forces/moments
   					\item Internal forces/moments
  				\end{itemize}
  				\begin{equation}
  					{n}_{(s)}^{'} = - {f}_{(s)}
  				\end{equation}
  				\begin{equation}
  					{m}_{(s)}^{'} = - {p}_{(s)}^{'} × {n}_{(s)} - {l}_{(s)}
  				\end{equation}
 			\end{itemize}
			\vspace{2cm}
			\column{0.5\textwidth}
			\begin{figure}[h]
				\centering
				\includegraphics[width=\textwidth]{images/equilibrium}
				\caption{Sections of the beam considered for the static equilibrium \cite{black_parallel_2018}.}
			\end{figure}
			\end{columns}
		\end{frame}

        \subsection{BVP}
        \begin{frame}
        	\frametitle{Boundary Value Problem}
			\begin{columns}
			\column{0.5\textwidth}
			\begin{itemize}%[<+->]
  				\item Constraints at the distal plate
  				\begin{itemize}%[<.->]
   					\item External wrench $ \Psi_{ext} = \begin{bmatrix} F \\ M \end{bmatrix}   					 $
   					\item Rod contribution $ \Psi_{i} = \begin{bmatrix} {n_i}_{(L_i)} \\ {m_i}_{(L_i)} \end{bmatrix}   					 $
  				\end{itemize}
  				\item Constraints at the base 
  				\begin{itemize}%[<.->]
   					\item Actuations $ \Psi_{a_i}$
   					\item Joints and geometry 
  				\end{itemize}
 			\end{itemize}
			\vspace{2cm}
			\column{0.5\textwidth}
			\begin{figure}[h]
				\centering
				\includegraphics[height=0.7\textheight]{images/BVP}
				\caption{Geometrical and actuation constraints for a Stewart-Gough CPR.}
			\end{figure}
			\end{columns}
		\end{frame}
	
	\section{Methods}	
		\subsection{Shooting_method_statics}
        \begin{frame}
        	\frametitle{Shooting Method in statics (Stewart-Gough CPR)}
        	\begin{columns}
			\column{0.5\textwidth}
			\begin{itemize}%[<+->]
  				\item ODE system in statics
  				\begin{itemize}
  					\item Equilibrium equations
  					\item Material properties
  					\item Geometrical considerations
  				\end{itemize}
  				\item Recursive solution
  				\begin{itemize}
  					\item Needs an intial guess
  					\item Evaluation on a cost function  
  					\begin{fleqn}
  					\begin{equation}
  						\textbf{\textit{f}} = 
  						\begin{bmatrix}
  							\sum{ {{n}_{i}}_{\left({L}_{i} \right)} }- F \\
  							\sum{ \left[{{p}_{i}}_{\left({L}_{i} \right)} × {{n}_{i}}_{\left({L}_{i} \right)} + {{m}_{i}}_{\left({L}_{i} \right)} \right] } - {p}_{d} × F-M \\
  							{p}_{d} + {R}_{d} {r}_{i} - {{p}_{i}}_{\left({L}_{i} \right)} \\
  							{\left[{{R}_{i}^{T}}_{\left({L}_{i} \right)} {R}_{d} - {{R}_{i}}_{\left({L}_{i} \right)} {R}_{d}^{T} \right]}^{V} 
  						\end{bmatrix}
  					\end{equation}\vfill
  					\end{fleqn}
  				\end{itemize}
 			\end{itemize}
 			%\vspace{2cm}
			\column{0.5\textwidth}
			\begin{figure}[h]
				\centering
				\includegraphics[height=0.7\textheight]{images/BVP}
				\caption{Geometrical and actuation constraints for a Stewart-Gough CPR.}
			\end{figure}
			\end{columns}
		\end{frame}
		
		\subsection{Shooting_method_dynamics}
        \begin{frame}
        	\frametitle{Shooting Method in dynamics}
			\begin{itemize}%[<+->]
				\item PDE system
				\begin{itemize}
					\item Derivative wrt to arc-lenght $ x^{'} = \diffp{x}{s} $
					\item Derivative wrt to time $\dot{x} = \diffp{x}{t} $
				\end{itemize}
				\item From PDE to ODE
				\begin{itemize}
					\item Implicit discretization
					\begin{equation}
						\diffp{x}{t} = {c}_{0} {x}^{(i)} + \sum_{k=1}^{\infty} {\left[{c}_{k} {x}^{\left(i-k \right)} + {d}_{k} {\dot{x}} ^ {\left(i-k \right)} \right]}
					\end{equation}
					\begin{equation}
						\diffp{x}{t} = {c}_{0} {x} ^ {\left(i \right)} + {c}_{1}^{\left(i-1 \right)} {x} ^ {\left(i-1 \right)} + {c}_{2}^{\left(i-2 \right)} {x} ^ {\left(i-2 \right)} + {d}_{1}^{\left(i-1 \right)} \diffp{x^{\left(i-1 \right)}}{t}
					\end{equation}
				\end{itemize}
 			\end{itemize}
		\end{frame}
		
		\subsection{Non linear solver}
        \begin{frame}
        	\frametitle{Non linear solver: Levenberg-Marquardt algorithm}
			\begin{itemize}%[<+->]
  				\item Iterative algorithm
  				\begin{itemize}
  					\item From initial conditions
  					\item Very sensitive
  				\end{itemize}
  				\item Evaluates influence of parameter vector $ u $
  				\begin{equation}
  				J = \diff{f}{u}
  				\end{equation}
  				\item Updates the parameter vector
  				\begin{equation}
  				{u}_{k+1} = {u}_{k} + {\left({J}_{k}^{T} {J}_{k} + \mu I \right)}^{-1} {J}_{k}^{T} {f}_{k}
  				\end{equation}
 			\end{itemize}
		\end{frame}
		
		\subsection{Strain_approach_intro}
        \begin{frame}
        	\frametitle{Strain approach}
			\begin{itemize}%[<+->]
  				\item Modelling of a continuum body in space
  				\begin{itemize}
  					\item Internally actuated Cosserat beam
  					\item In its dynamics
  					\begin{align}
  						\eta = \left( T^{-1}\dot{T} \right)^V && \xi = \left( T^{-1} {T}^{'} \right)^V
  					\end{align}
				\end{itemize}
  				\item From assumption on rod deformation
  				\begin{itemize}
  					\item Allowed $ \xi_a $, prohibited $ \xi_c $ twists
  					\item Strain generalized coordinates $ q_{[n\times1]}$
  				\end{itemize}
  				\item Define its dynamic model
 			\end{itemize}
		\end{frame}
		
		\subsection{Strain_approach_deteails}
        \begin{frame}
        	\frametitle{Strain approach, Lagrangian model of continuum manipulator}
			\begin{itemize}%[<+->]
  				\item Lagrangian model of continuum
  				\begin{fleqn}
  				\begin{equation}
  					\begin{bmatrix}
  							0 		\\
  						{Q}_{ad}	\\
					\end{bmatrix} 
					= 
					\begin{bmatrix}
						{M}_{0}		& {M}_{0\epsilon} 				\\ 
						{M}_{\epsilon0}	& {M}_{\epsilon\epsilon}	\\
					\end{bmatrix}
					\begin{bmatrix}
						{\dot{\eta}}_{0} 	\\ 
						{\ddot{q}}_{(t)}	\\
					\end{bmatrix}	
					+
					\begin{bmatrix}
						{{F}_{v}}_{\left(q , \dot{q} , {\eta}_{0} \right)}	\\
						{{Q}_{v}}_{\left(q , \dot{q} , {\eta}_{0} \right)}	\\
					\end{bmatrix}
					+
					\begin{bmatrix}
						{{F}_{c}}_{\left(q , {g}_{0} \right)}	\\ 
						{{Q}_{c}}_{\left(q , {g}_{0} \right)}	\\
					\end{bmatrix}
					+
					\begin{bmatrix}
						0 \\
						{K}_{\epsilon\epsilon} {q}_{\left(t \right)} + {D}_{\epsilon\epsilon} {\dot{q}}_{\left(t \right)}	\\
					\end{bmatrix}
  				\end{equation} 
  				\end{fleqn}
  				\item Virtual serial mechanism analogy			
  				\begin{itemize}
  					\item The Lagrangian model
  					\begin{equation}
  						\begin{bmatrix}
  						{F}_{0} \\
  						{Q}_{a}	\\
					\end{bmatrix} 
					= 
					\begin{bmatrix}
						{M}_{0}		& {M}_{0\epsilon} 				\\ 
						{M}_{\epsilon0}	& {M}_{\epsilon\epsilon}	\\
					\end{bmatrix}
					\begin{bmatrix}
						{\dot{\eta}}_{0} 	\\ 
						{\ddot{q}}_{(t)}	\\
					\end{bmatrix}	
					+
					\begin{bmatrix}
						{{F}_{v}}_{\left(q , \dot{q} , {\eta}_{0} \right)}	\\
						{{Q}_{v}}_{\left(q , \dot{q} , {\eta}_{0} \right)}	\\
					\end{bmatrix}
					+
					\begin{bmatrix}
						{{F}_{c}}_{\left(q , {g}_{0} \right)}	\\ 
						{{Q}_{c}}_{\left(q , {g}_{0} \right)}	\\
					\end{bmatrix}
  					\end{equation}
  					\item Recursive reconstruction
  				\end{itemize}
 			\end{itemize}
		\end{frame}
		
		\subsection{ICA_intro}
        \begin{frame}
        	\frametitle{Introduction to the Isogeometric Collocation Analysis}
			\begin{columns}
			\column{0.5\textwidth}			
			\begin{itemize}%[<+->]
  				\item NURBS curves represent vector field
  				\begin{itemize}
  					\item Control point as degree of freedom
  					\item Basis functions relate influence
  				\end{itemize}
  				\item Cost function
  				\begin{itemize}
  					\item Equilibrium equation evaluated at collocation points
  				\end{itemize}
 			\end{itemize}
 			%\vspace{2cm}
			\column{0.5\textwidth}
			\begin{figure}[h]
				\centering
				\includegraphics[width=\textwidth]{images/ICA}
				\caption{Rod centerline position and orientation represented with NURBS curves \cite{piegl_nurbs_1997}.}
			\end{figure}
			\end{columns}
		\end{frame}
		
		\subsection{ICA_details}
        \begin{frame}
        	\frametitle{Properties of the Isogeometric Collocation Analysis}
        	\begin{columns}
			\column{0.5\textwidth}
			\begin{itemize}%[<+->]
  				\item Less integrations
  				\begin{itemize}
  					\item In statics no integration
  					\item ODE in dynamics
  				\end{itemize}
  				\item Introduces possibility of modelling
  				\begin{itemize}
  					\item Contact between rods
  					\item Changes in shape and or material 
  					\item Rods coupling
  				\end{itemize}
 			\end{itemize}
 			\vspace{2cm}
			\column{0.5\textwidth}
			\begin{figure}[h]
				\centering
				\includegraphics[width=\textwidth]{images/AV_TV}
				\caption{Rod with properties changing a) axially and b) transversally}
			\end{figure}
			\end{columns}
		\end{frame}

  	\section{Conclusions}
      	\subsection{Conclusions}
        \begin{frame}
        	\frametitle{Conclusions}
        	\begin{itemize}
        		\item Types of robots
        		\begin{itemize}
        			\item Serial robots
        			\item Parallel robots
        			\item Continuum parallel robots
        		\end{itemize}
        		\item Model of deformable links
       		 	\begin{itemize}
        			\item Geometric
        			\item Equilibrium
        			\item Assembly
        		\end{itemize}
        		\item Methods
        		\begin{itemize}
        			\item Shooting Method
       		 		\item Strain Approach
       		 		\item Isogeometric Collocation Method
        		\end{itemize}
        	\end{itemize}
        \end{frame}
        	
      	\subsection{Expected_work}
        \begin{frame}
        	\frametitle{Expected work}
        	\begin{itemize}
        		\item A Gazebo Simulator for Continuum Parallel Robots
        		\item Immediate terms: Method(s) selection
        		\begin{itemize}
        			\item Previusly presented 
        			\item One or combination
        		\end{itemize}
        		\item Short terms: Solve the modelling
  				\begin{itemize}%[<.->]
   					\item Rod statics
        		\end{itemize}
        		\item Long terms
  				\begin{itemize}%[<.->]
   					\item Robot assembly
   					\item Visual interface
   					\item Robot dynamics
  				\end{itemize}
        	\end{itemize}
        \end{frame}
        	
        \subsection{Thanks}
        \begin{frame}
  			\centering \Large
  			\emph{Thank you for your attention}
        \end{frame}
        
\begin{frame}
	\printbibliography
\end{frame}
        	
\end{document}

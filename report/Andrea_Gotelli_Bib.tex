\documentclass{thesisreport}


\begin{document}

 \thispagestyle{empty}

\def\lskip{\vspace{0.5cm}}


\begin{tabular}{p{7cm}p{8cm}}
ÉCOLE CENTRALE DE NANTES
&
% EMARO students only
% \raggedleft FIRST YEAR INSTITUTION	
\end{tabular}

\vspace{2cm}

% CORO-IMARO students
%\begin{center} \large\sc MASTER CORO-IMARO\\ \normalsize{``CONTROL and ROBOTICS''} \end{center}

% EMARO students
\begin{center} \large\sc MASTER ERASMUS MUNDUS \\ \normalsize{EMARO+ ``European Master in Advanced Robotics''} \end{center}


\begin{center}
	2016 / 2017\\
	\lskip
	Master Thesis Report % or bibliography report
	\lskip
	
	Presented by \lskip 
	
	Student Name \lskip
	
	On Date \lskip\lskip
	
	{\Large \textbf{The title of the master thesis}}
	
	\vfill

Jury \lskip
		
	\end{center}
	


\begin{tabular}{p{3cm}p{7cm}p{5cm} }
 % President: & Name & Position (Institution) \\ & & \\     % for final defense only (not bibliography)
 Evaluators: & Name & Position (Institution) \\
	      & Name & Position (Institution) \\ 
	      & Name & Position (Institution) \\ & & \\  & & \\ 
  Supervisor(s):  & Name & Position (Institution) \\
		  & Name & Position (Institution) \\
% EMARO students only
%(EMARO)  & Co-supervisor from M1 & Position, M1 institution 
\end{tabular}

\lskip

\begin{flushleft}
 Laboratory: Laboratoire des Sciences du Numérique de Nantes LS2N
\end{flushleft}

\newpage
\thispagestyle{empty}
\null
\newpage
\addtocounter{page}{-1}
\pagestyle{fancy}
  
 
  \section*{Abstract}
 Continuum parallel robots are inspired by biological features like tentacles, trunks and snakes. Thanks to their design, continuum parallel robots show great performances in many applications. They can perform precise motions even in complex environments. 
 
 This report contains a review of the modelling for these robots. Moreover, we investigated some approaches to obtain a solution of their modelling. 
 
 
 
 
 
 \newpage
 
 \section*{Acknowledgements}
 I would like to thank first Dr. Sébastien Briot and professor Olivier Kermorgant for having developed this thesis proposal accordingly to my willing, and thus giving me the opportunity to work under their stimulating supervision. 
 
  \section*{Abbreviations}
 \begin{tabular}{cp{0.8\textwidth}}
 	\textbf{CPR}	&	Continuum Parallel Robot			\\
 	\textbf{ODE}	&	Odinary Differential Equation		\\
 	\textbf{PDE}	&	Partial Differential Equation		\\
 	\textbf{BVP}	&	Boundary Value Problem				\\
 	\textbf{ICA}	&	Isogeometric Collocation Analysis	\\
 	\textbf{NURBS}	&	Non-Uniform Rational B-Splines		\\
 	\textbf{DGM}	&	Direct Geometric Model				\\
 	\textbf{IGM}	&	Inverse Geometric Model				\\
 	\textbf{DDM}	&	Direc Dynamic Model					\\
 	\textbf{IDM}	&	Inverse Dynamic Model				\\
 \end{tabular}

 \tableofcontents
 
 
 \chapter*{Introduction}
 \addcontentsline{toc}{chapter}{Introduction}	 % non-numbered chapters do not appear in table of contents by default
 \section{Motivations for CPRs}
  Nowadays robots are used in an increasing number of applications, such as colonscopy \cite{black_modeling_2017} and teleoperations \cite{till_efficient_2015}.
 As far as the robot architecture is concerned, we can distinguish between serial and parallel arrangements.
 
 The former are the simpler and they are widely used thanks to their structure. Serial robots are composed of links connected in series by joints. Their inertia plays an important role in their design due to the weight of the actuators mounted on the arm links. The links weight is considerably high because they must be stiff enough to sustain the whole structure.

 On the other hand, parallel robots have a closed loop shape architecture. Some architecture designs allow to place the actuators in the base only. Consequently, parallel manipulators design results in stiffer and low inertia architectures, compared to serial robots ones. However, the multiple joints involved require small assembly tolerances and bring additional maintenance costs.
 
 Concerning continuum parallel robots, they consist in an assembly of slender beams, or rods, connected to a platform. These rods replace the rigid links propers of standard manipulators and are flexible components, assumed as highly deformable. In fact, the motion of these robots is due to the deformation of their beams assembly.
 The platforms are placed at least on the top and bottom of the robot, providing place for the end effector and actuators respectively. However, intermediate disks can be placed in the robot body providing structural constraints for the beams.  
 
 This assembly brings several advantages to the robot performances. First the lightweight body of the robot has a smaller inertia, compared to the other kinds of robots, resulting in a more precise and less energy-consuming movements. Moreover, the robot assembly results easier as less joints are involved in the design. Additionally, the high deformabilty of the rods may enhance the robot safety in presence of humans. In fact, if a human happens to be in the robot trajectory, the rod hitting the human could bend, reducing the possible human damage. Finally, these robots have the possibility to be mignaturized. A miniaturization is beneficial where small and precise operations are required, such as in a surgery or other medical applications \cite{black_modeling_2017}. 
 
 On the other hand, these robots have limited payload compared to the serial and parallel ones. They also show more instability configurations.
 
 In summary, continuum parallel robots are very reliable, safe and versatile. The main difficulties come with their modelling. Their modelling does not have an analitycal solution and thus it as to be computed numerically in each position and configuration. In the literature there are several methods proposed to solve this problem. In the following, the investigated methods are presented and discussed.
 
 \section{Aim of this thesis}
 The aim of this master thesis is to investigate and develop a general simulator for continuum parallel robots. The simulator must be able to handle different kinds of architecture for CPRs, including change in size of some components of the robot as well as different arrangements for the robot links \cite{bryson_toward_2014}. In the literature, two assemblies are usually considered. The first is the Stewart-Gough CPR. Its static modelling has been investigated by Till et al \cite{till_efficient_2015}\cite{black_parallel_2018}, who also developed the dynamic modelling \cite{till_real-time_2019}. They also investigated the second common assembly, consisting in tendon-driven robot, as well as a fluidic soft robot \cite{till_real-time_2019}. These different robots require modelling with specific care to individual features, to better represent their characteristic behavior. 
 
 For the graphical interface, the simulator will rely on Gazebo \cite{koenig_design_2004}\cite{noauthor_gazebo_nodate}, a widely use physical simulator in the ROS framework \cite{noauthor_rosorg_nodate}. Practically speaking, the simulator consists in the development of some plugins for Gazebo.
 These plugins implement some methods to model the robot behavior. Once the model is obtained they use it to solve one of the problems: DGM, IGM, DDM, IDM.
  The modelling problem corresponds to find the actuator positions and wrenches, which are the inputs, to obtain the desired output. In the DGM case, we want to obtain the end effector position given the input. In the IGM case, we want to calculate the input in order to obtain a given output: position or wrench in the end effector. On the other hand, when we consider the dynamics of the robot, when computing the input/output relation, the dynamic of the robot plays an important role. 
 
 In order to correctly simulate the robot, the simulator must solve the modelling for different robot architectures. In the literature, some models have already been investigated. Black et al developed the DGM for a Stewart-Gough robot \cite{black_parallel_2018}. For the same robot, Till et al presented the solution for the IGM \cite{till_real-time_2019} and IDM \cite{till_efficient_2015}. Morevoer, other ad oc solution are presented in the literature \cite{till_efficient_2015}\cite{black_parallel_2018}\cite{till_real-time_2019}. However, these approaches are well suited only for a specific CPR architecture, with a poor prospective of generalization. This master thesis aims to define a general simulator which can be used to simulate these robots indipendently by they structure. 
 
 \chapter{State of the art}
 Generally speaking, in order to identify the equilibrium configuration of a CPR, we are facing to the solution of a static or a dynamic problem. This solution corresponds to the configuration at equilibrium under the external load, the actuators action and the assembly constraints. To model each rod the Cosserat theory is widely used in literature, as in the work of Black \cite{black_modeling_2017}\cite{black_parallel_2018} and Till \cite{till_efficient_2015}\cite{till_real-time_2019}. In the Cosserat theory, each rod is described geometrically by its centerline position and orientation. The rod is reduced to a 1D body where, for every coordinate in the arc-lenght, there is a corresponding cross section position and orientation. The centerline positions and cross section orentations are defined with respect to a reference frame external to the rod body, expressing their coordinates in an Euclidean space \cite{selig_geometric_2005}.
  
 At each cross section, equilibrium considerations correlate punctual or distributed external forces to internal actions. The rod deformation is described by equations accounting for the influences of these forces, material properties and geometric properties. These equations define a ODE system. This system solution depends on the boundary and initial conditions. The boundary conditions come from the robot architecture, at the platform and at the base, where the assembly and the applied loads bring contrains to the system solution. In fact, the joints implemented and their position, with respect to the end effector frame, play an important role in the BVP, as explained by Black et al \cite{black_parallel_2018}. Depending on the joint type, one or more components of the internal forces must be null or assume a value directly depending on the external wrench applied on or by the end effector.  
 
 In the following, three approaches that have been investigated, in accordance with the literature, are briefly presented.
 
 \section{Shooting Method}
 The shooting method is a generic numerical tool for the solution of a boundary value problem and it is widely used in the literature \cite{black_modeling_2017}\cite{black_parallel_2018}\cite{florian_geometrically_2020}. This approach allows us to simulate CPRs with high precision, as required for surgical operations, like the one described by Black et al \cite{black_modeling_2017}. Moreover, it is possible to evaluate robot performances in simulation as Black et al explained \cite{black_parallel_2018} and, with greater details \cite{black_modeling_2017}. The shooting method is an iterative process, based on a non linear solver like the Levenberg-Marquardt algorithm, which is widely discussed in literature: an introduction of concepts and properties is presented \cite{lourakis_brief_nodate}.
  
 The solver is used for minimizing the value of a cost function \cite{florian_geometrically_2020}. In the CPRs case, the cost function is defined as a row vector which contains the equilibrium equations, the geometrical constrains and the material relations with respect to the loads. In the equilibrium configuration, the cost function is, theoretically, equal to zero, as the solver converged to a solution of the problem.
 
 Looking in detail at the shooting method, it starts with the choice of an intial guess. From these initial conditions, the method evaluates the Jacobian relationship between the cost function and the unknown variables, that consist of the solution for the current configuration. Using this Jacobian relation, the non linear solver computes another guess. This iterative process stops when it converges to an equilibrium solution or when it reaches maximum number of iterations. For computing the Jacobian relationship, the process can be splitted into different threads. As discussed by Till et al, the procedure can be parallelized, obtaining appreaciable improvements in terms of computational time \cite{till_efficient_2015}.  
 
 When considering the dynamics modelling, the complexity increases. The equilibrium equations and the BVP have to take into account the dynamic of an assembly of sclender elestic rods. In the dynamics case, quantities like the centerline position and cross section orientation are depending both on the arc-lenght coordinate and time. Similarly, all the other vector fields are funtion of time and thus their time derivative must be considered. 
 
 The system of equation becomes a PDE one, having derivatives with respect to the arc-lenght and time. Till et al proposed a solution to approximate the time derivative with an implicit time differentiation formula \cite{till_real-time_2019}. It consists in replacing the time derivative by a sum of the current value and the history of previous values. This allows an intuitive application of the shooting method, as it will change only the current guess. 
 
 A critical aspect of this solution is the choice of the timestep. Accordingly to Till et al, too small timestep will bring high numerical cancellations \cite{till_real-time_2019}. On the other hand, timestep of a too big size lead to instability. 
 
 This method is able to find non stable configurations and to model the robot reaching the closer stable configuration. However, it is very sensible to the initial conditions. This sensitivity may lead to problem when, for slightly different initial conditions, the final solutions are different. 
 
 
 \section{Strain Parametrization}
 
 For the description of the strain parametrization approach, we follow the work of Boyer et al \cite{boyer_dynamics_2019}. They developed a modelling for a slender body deforming and moving in space, treated as an internally actuated Cosserat beam.
 
 The strain parametrization approach integrates the Cosserat theory with the notion of allowed and prohibited twists. These twists are defined in accordance to the material law and the possible motion of the beam which depend on the assumption on its deformation. Being justified by the theory of slender bodies, we could neglect the shear deformations.
 
 The configuration space of the floating beam is the composition of the configuration space of the beam base frame and the configuration space of its shape. This last space is a functional one: an infinite dimensional space for the parametrized curve. 
 This formulation is relaxed by the introduction of the allowed motions, their number limits the dimension of the beam shape configuration space. 
 The composition of the two configuration spaces is then discretized using a matrix of basis functions and a vector of strain generalized coordinates. 
 
 Boyer et al developed the Lagrangian model for this floating beam, which results in a set of equations similar to the ones proper of riged-links robots \cite{boyer_dynamics_2019}. In order to use the model on the CPR, all the terms have to be computed numerically at each timestep. They propose to obtain the CPR Lagrangian model starting from a similar one, but applied to a virtual serial mechanism. They use one of the algorithm proposed by Walker et al \cite{walker_efficient_1982}. The idea is to reconstruct all the matrices of the model with an interative process. To this end, algorithms for computing the forward and inverse dynamics are presented and detailed in Featherstone's book \cite{featherstone_rigid_2008}.

 
 \section{Isogeometric Collocation Analysis}
 The isogeometric collocation analysis is another generic numerical tool which can solve the BVP. One of the ICA properties is that it allows to find the equilibrium configurations without numerically integrate the rod in its arc-lenght. 
 It relies on the properties of NURBS curves \cite{piegl_nurbs_1997}. NURBS curves can intuitively represent any vector field such as position or orientation in space. These curves are defined with a set of control points, that are points in the vector space, some breakpoints and basis functions depending on the coordinate along the curve.
 
  The breackpoints are a set of points called knots that are placed along the curve and they define the range of activation for each basis functions. The basis functions link the influence of each control points to the curve as a function of the position along the curve itself. The basis functions are recursively computed starting with zero degree polynomials. These polynomials have constant value in their range of definition and they are zero outside from it. Basis functions of greater order are defined on top on previous ones. 
  
 As explained in the book of Piegl, these curves have properties such as differentiabily and affine transformations \cite{piegl_nurbs_1997}. The differentiability is ensured by the continuity of the higher order basis functions while affine transformations to the curve can be obtained applying the same transformations on the control points.
 
 As discussed, NURBS curves can describe any vector field. In literature, they are used to describe the centerline position and cross-section orientation of a slender body \cite{weeger_isogeometric_2017-1}. In order to account for the equlibrium equations, the authors define new points along the curve: the collocation points. At these points, the equilibrium equations and the constrains are evaluated. As a result, they obtain a cost function that requires a non-linear solver for the identification of the unknown control points.
 
 Additional features can be considered with this approach. It is possible to determine frictionless contact, and its effects, as presented by Weeger et al either between two different rods or a rod with itself \cite{weeger_isogeometric_2017}. Later, they extended their studies on contacts in the frictional case \cite{weeger_isogeometric_2018}. They also presented a modelling approach for varying properties of the rod \cite{weeger_fully_2018}. Material and geometrical properties may vary along the rod axis or transversally in the rod section. Moreover, they proposed a modelling approach for the rods coupling \cite{weeger_isogeometric_2017-1}.
 
 When it comes to deal with the rod dynamics, the method requires integration, defining thus an ODE system. The time derivative requires a discretization system, Weeger et al propose the Crank-Nicolson model, proving its performances in numerical applications \cite{weeger_isogeometric_2018}. 


 \chapter*{Conclusion}
 \addcontentsline{toc}{chapter}{Conclusion}
 
 In this report, some of the main modelling approaches were briefly presented. We discussed how the CPRs can be modeled and which are the problems that have to be addressed. Three methods were presented as possible approaches to solve the problems. 
 
 
 Having investigated these methods, it is necessary to define an approach leading to the realization of the simulator. After having established the method or the combination of methods to implement, the simulation of a single static rod will be addressed. 
 
 

 \addcontentsline{toc}{chapter}{Bibliography}
 
 \bibliography{../biblio}
 
 
 
 
\end{document}
